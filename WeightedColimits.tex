\documentclass[11pt]{amsart}          
\usepackage{amsfonts}
\usepackage{amssymb}  
\usepackage{amsthm} 
\usepackage{amsmath} 
\usepackage{tikz-cd}
\usepackage{float}
\usepackage[]{hyperref}
\hypersetup{
  colorlinks,
  linkcolor=blue,
  citecolor=blue,
  urlcolor=blue}

\begin{document}

It's funny how the same ideas repeat themselves in different branches of mathematics. Calculus, for instance, is built around metric spaces (or, more generally, Banach spaces) and measures. A limit of a sequence is defined by points getting closer and closer together. An integral is an area under a curve. In category theory, though, we don't talk about distances or areas (except for Lawvere's take on metric spaces), and yet we have the abstract notion of a \href{https://bartoszmilewski.com/2015/04/15/limits-and-colimits/}{limit}, and we use integral notation for \href{https://bartoszmilewski.com/2017/03/29/ends-and-coends/}{ends}. The similarities are uncanny. This blog post was inspired by me trying to understand the analogy between a coend formula and a multi-dimensional integral. On the one hand we have a coend over a Grothendieck fibration, which is like a two-dimensional category, on the other hand we have a coend over its base category. To understand it, though, I had to get a better handle on weighted colimits which, as it turns out, are even more general than Kan extensions. 

\section{Category of elements as coend}

\href{https://bartoszmilewski.com/2019/10/09/fibrations-cleavages-and-lenses/}{Grothendieck fibration} is like splitting a category in two orthogonal directions, the base and the fiber. Fiber may vary from object to object (as in dependent types, which are indeed modeled as fibrations). 

The simplest example of a Grothendieck fibration is the category of elements, in which fibers are simply sets. Of course, a set is trivially a category: a discrete category with no morphisms between elements (except for compulsory identity morphisms). A category of elements is built from a category $C$ using a functor
\[F \colon C \to Set\]
Such a functor is traditionally called a presheaf. Objects in the category of elements are pairs $(c, x)$ where $c$ is an object in $C$, and $x \in F c$. A morphism from $(c, x)$ to $(c', x')$ is a morphism  $f \colon c \to c'$ in $C$, such that $(F f) x = x'$
There is an obvious projection functor that forgets the second component of the pair
\[\Pi \colon (c, x) \mapsto x\]

Sometimes, the category of elements is written using integral notation, since objects of this category form a disjoint union (sum) of sets $F c$
\[\int^{c \colon C} F c\]
but that conflicts with the colimit notation so, following \href{https://arxiv.org/abs/1501.02503}{Fosco Loregian}, I'll use the notation
\[C\int^{c \colon C} F c\]
An interesting example is a comma category. It's the category $d/K$, with $K \colon D \to C$, whose elements are pairs $(c, f)$ with $f \colon c \to K d$
of elements for the functor $F c = D(d, K c)$
\[d/K \cong C\int^c D(d, K c)\]
\section{Weighted colimit as coend}
Colimit is a universal cocone over a diagram. A diagram is a selection of objects and morphisms in $C$ given by a functor $D \colon J \to C$ from some indexing category $J$. The sides of a cocone are a selection of morphisms from the vertices of the diagram to the apex $c$ of the cocone. For any given indexing object $j \colon J$, we select an element of the hom-set $C(D j, c)$, as a \emph{wire} of the cocone. This selection can be described as a function from the singleton set $*$ or, equivalently, a member of $Set(*, C(D j, c))$. In fact, we can describe a cocone as a natural transformation from the constant functor $1 \colon j \mapsto *$
\[ [J^{op}, Set](1, C(D -, c))\]
and get all the cocone commuting conditions from naturality. (Here, $[J^{op}, Set]$ is the category of presheaves, or functors from $J^{op}$ to $Set$, with natural transformations as morphisms.)

This kind of razor-sharp selection using singleton sets doesn't generalize very well to enriched categories. Cocones are built from zero-thickness wires. What we need instead is what \href{http://www.tac.mta.ca/tac/reprints/articles/10/tr10.pdf}{Max Kelly} calls \emph{cylinders} obtained by replacing the constant functor $1^{op}\colon J \to Set$ with a more general functor $W \colon J^{op} \to Set$. The result is that we get a \emph{weighted} colimit (or and indexed colimit, as Kelly calls it), $\mbox{colim}^W D$, as the universal cocone defined by 
\[ [J^{op}, Set](W-, C(D -, c))\]
Universality can be expressed as a natural isomorphism
\[[J^{op}, Set](W-, C(D -, c))  \cong  C(\mbox{colim}^W D, c)\]
We interpret this as a one-to-one correspondence: for every weighted cocone with the apex $c$ there is a unique morphism from the colimit to the apex $c$. Naturality conditions guarantee that the appropriate triangles commute.

A weighted colimit can be expressed as a coend
\[\mbox{colim}^W D \cong \int^{j \colon J} W j \cdot D j\]
The dot here stands for the tensor product of a set by an object defined by the formula
\[C(s \cdot c, c') \cong Set(s, C(c, c'))\]
If you think of $s \cdot c$ as the sum of $s$ copies of the object $c$, then the above asserts that the set of functions from a sum (coproduct) is equivalent to a product of functions, one per element of the set $s$,
\[(\coprod_s c) \to c' \cong \prod_s (c \to c')\]
 
The derivation of the coend formula follows the standard coend-fu and the Yoneda lemma:
\begin{align*}
C(\mbox{colim}^W D, c) & \cong [J^{op}, Set]\big(W-, C(D -, c)\big) \\
 &\cong \int_j Set \big(W j, C(D j, c)\big) \\
 &\cong \int_j C(W j \cdot D j, c) \\
 &\cong C(\int^j W j \cdot D j, c)
\end{align*}

\section{Two dimensional integral as a coend}


A coend is like an integral. We can integrate a functor over a fibration. If the functor is constant along the fiber, we can integrate in that direction. Here, Pi is the projection to base.

Since fibers depends on b, different values of D b end up weighted differently. The result is an integral (coend) over the base, with values of D b weighted by sets W b (the dot is the tensor product of a set and an object).

Using more traditional notation, this is the formula that relates a (conical) colimit over the category of elements and a weighted colimit.

\[\underset{J \int W}{\mbox{colim}} \; D \circ \Pi  \cong \mbox{colim}^W D\]

\[\int^{(j, x) \colon J\int W} (D \circ \Pi) (j, x) \cong \int^{(j, x) \colon J\int W} D j  \cong   \int^{j \colon J} W j \cdot D j\]

\section{misc}

\begin{align*}
\Pi &\colon B \int W  \to B \\
\Pi &\colon (b, x) \mapsto b
\end{align*}

\[\underset{B \int W}{\mbox{colim}} \; D \circ \Pi  \cong \mbox{colim}^W D\]

\[\int^{(b, x) \colon B\int W} (D \circ \Pi) (b, x) \cong \int^{(b, x) \colon B\int W} D b  \cong   \int^{b \colon B} W b \cdot D b\]


\end{document}