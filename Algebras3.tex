\documentclass[11pt]{amsart}          
\usepackage[a4paper,verbose]{geometry}
\geometry{top=3cm,bottom=3cm,left=3cm,right=3cm,textheight=595pt}
\setlength{\parskip}{0.3em}
% ==============================
% PACKAGES
% ==============================

\usepackage{amsfonts}
\usepackage{amssymb}  
\usepackage{amsthm} 
\usepackage{amsmath} 
\usepackage{caption}
\usepackage[inline]{enumitem}
\setlist{itemsep=0em, topsep=0em, parsep=0em}
\setlist[enumerate]{label=(\alph*)}
\usepackage{etoolbox}
\usepackage{stmaryrd} 
\usepackage[dvipsnames]{xcolor}
\usepackage[]{hyperref}
\hypersetup{
  colorlinks,
  linkcolor=blue,
  citecolor=blue,
  urlcolor=blue}
\usepackage{graphicx}
\graphicspath{{assets/}}
\usepackage{mathtools}

\usepackage{tikz-cd}
\usepackage{minted}
\usepackage{float}
\usetikzlibrary{
  matrix,
  arrows,
  shapes
}

\setcounter{tocdepth}{1} % Sets depth for table of contents. 

% ======================================
% MACROS
%
% Add your own macros below here
% ======================================

\newcommand{\rr}{{\mathbb{R}}}
\newcommand{\nn}{{\mathbb{N}}}
\newcommand{\iso}{\cong}
\newcommand{\too}{\longrightarrow}
\newcommand{\tto}{\rightrightarrows}
\newcommand{\To}[1]{\xrightarrow{#1}}
\newcommand{\Too}[1]{\To{\;\;#1\;\;}}
\newcommand{\from}{\leftarrow}
\newcommand{\From}[1]{\xleftarrow{#1}}
\newcommand{\Cat}[1]{\mathbf{#1}}
\newcommand{\cat}[1]{\mathcal{#1}}
\newtheorem*{remark}{Remark}
\renewcommand{\ss}{\subseteq}
\newcommand{\hask}[1]{\mintinline{Haskell}{#1}}
\newenvironment{haskell}
  {\VerbatimEnvironment
  	\begin{minted}[escapeinside=??, mathescape=true,frame=single, framesep=5pt, tabsize=1]{Haskell}}
  {\end{minted}}

% ======================================
% FRONT MATTER
% ======================================

\author{Bartosz Milewski}
\title{Fixed Points of Endofunctors}

\begin{document}

\section{Initial algebra and catamorphism}

The initial algebra can be defined by its mapping out property.

\begin{haskell}
newtype Mu f = Mu (forall a. (f a -> a) -> a)
\end{haskell}
Notice that this definition requires the following language pragma
\begin{haskell}
{-# language RankNTypes #-}
\end{haskell}
This definition works because for every least fixed point one can define a catamorphism, which can be rewritten as
\begin{haskell}
cata :: Functor f => Fix f -> (forall a . (f a -> a) -> a)
cata (Fix x) = \alg -> alg (fmap (flip cata alg) x)
\end{haskell}
(\hask{flip} is a function that reverses the order of arguments of its (function) argument.) What the definition of \hask{Mu} is saying is that it's an object that, for all algebras, has a mapping out to a catamorphism.

It's easy to define a catamorphism in terms of \hask{Mu}, since \hask{Mu} \emph{is} a catamorphism
\begin{haskell}
cataMu :: Functor f => Algebra f a -> Mu f -> a
cataMu alg (Mu cata) = cata alg
\end{haskell}
The challenge is to construct terms of type \hask{Mu f}. For instance, let's convert a list of \hask{a} to a term of type \hask{Mu (ListF a)}
\begin{haskell}
mkList :: forall a. [a] -> Mu (ListF a)
mkList as = Mu cata
  where cata :: forall x. (ListF a x -> x) -> x
        cata unf = go as
          where
            go [] = unf NilF
            go (n: ns) = unf (ConsF n (go ns))
\end{haskell}
Notice that we use the type \hask{a} defined in the type signature of \hask{mkList} to define the type signature of the helper function \hask{cata}. For the compiler to identify the two, we have to use the pragma
\begin{haskell}
{-# language ScopedTypeVariables #-}
\end{haskell}
You can now verify that
\begin{haskell}
cataMu sumAlg (mkList [1..10])
\end{haskell}
produces the correct result for the following algebra
\begin{haskell}
sumAlg :: Algebra (ListF Int) Int
sumAlg NilF = 0
sumAlg (ConsF a x) = a + x
\end{haskell}

\section{Terminal coalgebra and anamorphism}

The terminal coalgebra, on the other hand, is defined by its mapping in property. This requires a definition in terms of existential types. If Haskell had an existential quantifier, we could write the following definition for the terminal coalgebra
\begin{haskell}
data Nu f =  Nu (exists a. (a -> f a, a))
\end{haskell}
Existential types can be encoded in Haskell using the so called Generalized Algebraic Data Types or GADTs
\begin{haskell}
data Nu f where 
  Nu :: (a -> f a) -> a -> Nu f
\end{haskell}
The use of GADTs requires the language pragma 
\begin{haskell}
{-# language GADTs #-}
\end{haskell}

The argument is that, for every greatest fixed point one can define an anamorphism

\begin{haskell}
ana :: Functor f => forall a. (a -> f a) -> a -> Fix f
ana coa x = Fix (fmap (ana coa) (coa x))
\end{haskell}
We can uncurry it
\begin{haskell}
ana :: Functor f => forall a. (a -> f a, a) -> Fix f
ana (coa, x) = Fix (fmap (curry ana coa) (coa x))
\end{haskell}
A universally quantified mapping out 
\begin{haskell}
forall a. ((a -> f a, a) -> Fix f)
\end{haskell}
is equivalent to a mapping out of an existential type (in pseudo-Haskell)
\begin{haskell}
(exists a. (a -> f a, a)) -> Fix f
\end{haskell}
which is the type signature of the constructor of \hask{Nu f}.

The intuition is that, if you want to implement a function from an existential type---a type which hides some other type \hask{a} to which you have no access---your function has to be prepared to handle any \hask{a}. In other words, it has to be polymorphic in \hask{a}. 

Since in an existential type we have no access to the hidden type, it has to provide both the ``producer'' and the ``consumer'' for this type. Here we are given a value of type \hask{a} on the produces side, and the function \hask{a -> f a} as the consumer. All we can do is to apply this function to \hask{a} and obtain the term of the type \hask{f a}. Since \hask{f} is a functor, we can lift our function and apply it again, to get something of the type \hask{f (f a)}. Continuing this process, we can obtain arbitrary powers of \hask{f} acting on \hask{a}. We get a recursive data type.

An anamorphism in terms of \hask{Nu} is given by
\begin{haskell}
anaNu :: Functor f => Coalgebra f a -> a -> Nu f
anaNu coa a = Nu coa a
\end{haskell}

Notice however that we cannot directly pass the result of \hask{anaNu} to \hask{cataMu} because we are no longer guaranteed that the initial algebra is the same as the terminal coalgebra for a given functor.

\section{End/Coend formulation}

Let's rewrite \hask{Mu} using GADTs
\begin{haskell}
data Mu f where
  Mu :: (forall a. (f a -> a) -> a) -> Mu f
\end{haskell}
We use a natural transformation to construct a \hask{Mu}. Categorically, we can write it as an end
\[\mu f = \int_a a^{C(f a, a)}\]
It's an end over the profunctor
\[p a b = b^{ C(f b, a)}\]
Where the power is defined as
\[ C(x, a^s) \cong Set(s, C(x, a)) \]

Projection from the end is a catamorphism
\[\pi_a \colon \mu f \to a^{C(f a, a)}\]
It's a morphism from the hom-set
\[C(\mu f, a^{C(f a, a)})\]
or an element of
\[Set(C(f a, a), C(\mu f, a))\]

Similarly, we can rewrite \hask{Nu}
\begin{haskell}
data Nu f where 
  Nu :: (a -> f a) -> a -> Nu f
\end{haskell}
as a coend
\[\nu f = \int^a C(a, f a) \cdot a \]
over the profunctor
\[p a b = C(a, f b) \cdot b\]
where the copower is defined as
\[ C(s \cdot a, x) \cong Set(s, C(a, x))\]

Injection into the coend is an anamorphism
\[ \iota_a \colon C(a, f a) \cdot a \to \nu f\]
It's a morphism from the hom-set
\[C(C(a, f a) \cdot a, \nu f)\]
or an element of
\[Set(C(a, f a), C(a, \nu f))\]

Because of Lambek's lemma, an initial algebra is also a coalgebra, and a terminal coalgebra is also an algebra. Universality, therefore, tells us that there is a unique algebra morphism (as well as a unique coalgebra morphism) 
\[\phi \colon \mu f \to \nu f\]
This is a canonical embedding, but not necessarily an isomorphism.

The two profunctors in the definition of \hask{Mu} and \hask{Nu} can be written as
\begin{haskell}
data M f a b = M ((f b -> a) -> b)
\end{haskell}

\begin{haskell}
instance Functor f => Profunctor (M f) where
  dimap g g' (M h) = M (\j -> g'( h (g . j . fmap g')))
\end{haskell}

\begin{haskell}
data N f a b = N (a -> f b) b
\end{haskell}

\begin{haskell}
instance Functor f => Profunctor (N f) where
  dimap g g' (N h b) = N (fmap g' . h . g) (g' b)
\end{haskell}

\section{Hylomorphism}

If the mapping from from the terminal coalgebra $\nu$ to the initial algebra $\mu$ exists, it is an element of the following hom-set

\[C \Big(\int^a C(a, f a) \cdot a, \int_b b^{C(f b, b)} \Big)\]

By co-continuity of the hom-functor, this is isomorphic to

\[ \int_a C \Big( C(a, f a) \cdot a, \int_b b^{C(f b, b)}\Big)\]

Using continuity we get

\[ \int_{a, b} C \Big( C(a, f a) \cdot a, b^{C(f b, b)}\Big)\]

Using the definition of the copower
\[ C(s \cdot a, x) \cong Set(s, C(a, x))\]
we get
\[ \int_{a, b} Set \Big(C(a, f a), C(a, b^{C(f b, b)})\Big)\]
And using the definition of the power
\[ C(x, a^s) \cong Set(s, C(x, a)) \]
we get
\[ \int_{a, b} Set \Big(C(a, f a), Set\big(C(f b, b), C(a, b)\big)\Big)\]
Finally, applying the currying adjunction we get
\[ \int_{a, b} Set \Big(C(a, f a) \times C(f b, b), C(a, b)\Big)\]
in which you may recognize a hylomorphism

\begin{haskell}
hylo :: Functor f => Coalgebra f a -> Algebra f b -> a -> b
hylo coa alg = alg . fmap (hylo coa alg) . coa
\end{haskell}

\section{Bibliography}
\begin{itemize}
  \item Michael Barr, \href{http://www.math.mcgill.ca/barr/papers/trmclg.pdf}{Terminal coalgebras for endofunctors on sets}
\end{itemize}

\section{Random things}
\subsection{Initial algebra structure map}
\[j \colon f (\mu f) \to \mu f\]
\[C\Big(f (\mu f), \int_b b^{C(f b, b)}\Big)\]
is a member of
\[\int_b C\Big(f (\mu f), b^{C(f b, b)}\Big)\]
Using the definition of the power
\[ C(x, a^s) \cong Set(s, C(x, a)) \]
we get
\[\int_b Set\Big(C(f b, b), C\big( f (\mu f), b\big) \Big)\]
Using Yoneda lemma we replace $b$ with $f (\mu f)$
\[C(f (f (\mu f)), f (\mu f)) \]

\subsection{Terminal coalgebra structure map}
\[k \colon \nu f \to f (\nu f)\]
is a member of
\[ C\Big( \int^a C(a, f a) \cdot a, f  (\nu f) \Big)\]
\[ \int_a C\Big( C(a, f a) \cdot a, f  (\nu f) \Big)\]
Using the definition of the copower
\[ C(s \cdot a, x) \cong Set(s, C(a, x))\]
we get
\[ \int_a Set\Big( C(a, f a), C\big( a, f  (\nu f) \big) \Big)\]
Yoneda lemma
\[ C(f(\nu f), f ( f (\nu f))) \]

\subsection{Kan extensions}
\[\mu f = \int_a a^{C(f a, a)}\]
\[\nu f = \int^a C(a, f a) \cdot a \]
These formulas are reminiscent of Kan extensions. 
For comparison, the right  Kan extension of $g$ along $f$ is given by
\[(Ran_f g) c = \int_a (g a)^{C(c, f a)}\]
The left Kan extension is
\[(Lan_f g) c = \int^a C(f a, c) \cdot g a \]
If $f$ has left and right adjoints, they are given by
\[Ran_f Id \dashv f \dashv Lan_f Id\]
In particular, using the adjunction
\[(Lan_f Id) c = \int^a C(a, (Lan_f Id) c) \cdot a\]
This shows that $(Lan_f Id) c$ is a fixed point of the functor
\[ \Phi(x) = \int^a C(a, x) \cdot a \]

\subsection{Ends as limits}

Twisted arrow category on $\textit{Tw}(\Cat C)$ has, as objects, morphisms in $\Cat C$ (or, strictly speaking, triples $(a, b, f \colon a \to b)$). A morphism from $f \colon a \to b$ to $g \colon a' \to b'$ is a pair of morphisms 
\[(h \colon a' \to a, h' \colon b \to b')\]
For every profunctor $p \colon C^{op} \times C \to \Cat{Set}$ define a functor $\bar p \colon \textit{Tw}(\Cat C) \to Set$. On objects
\[\bar p (a, b, f) = p a b\]
and on morphisms, it's just profunctor lifting.

It can be shown that the end is just a limit over the twisted arrow category
\[\int_c p c c \cong \lim_{Tw(C)} \bar p\]
Similarly, the coend is a colimit over $Tw(C^{op} )^{op}$
\[\int^c p c c \cong \underset {Tw(C^{op})^{op}}  { \mbox{colim}} \bar p\]

\subsection{Iterative solution}
Terminal coalgebra is a limit, and initial algebra is a colimit of these two chains
\begin{figure}[H]
\centering
\begin{tikzcd}
  f (\nu f)
  \\
  \nu f
  \arrow[u, "k"]
  \arrow[d, "\pi_1"]
  \arrow[dr, "\pi_{(f 1)}"]
  \arrow[drr, bend left, "\pi_{(f^2 1)}"]
  \\
  1
  & f 1
  \arrow[l, "\mbox{!`}"]
  & f^2 1
  \arrow[l, "f \mbox{!`}"]
  & ...
  \arrow[l, dashed]
   \\
  0
   \arrow[u, "!"]
  \arrow[r, "!"]
  \arrow[d, ""]
   \arrow[d, "\iota_0"]
   &f 0
   \arrow[u, "f !"]
   \arrow[r, "f !"]
   \arrow[dl, "\iota_{(f 0)}"]
   &f^2 0
   \arrow[u, "f^2 !"]
   \arrow[r, dashed]
   \arrow[dll, bend left, "\iota_{(f^2 0)}"]
   & ...
   \\
   \mu f
   \arrow[uuu, bend left, "\varphi"]
   \\ 
   f (\mu f)
   \arrow[u, "j"]
\end{tikzcd}
\end{figure}




\end{document}
\maketitle{}