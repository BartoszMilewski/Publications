\documentclass[11pt]{amsart}          
\usepackage{amsfonts}
\usepackage{amssymb}  
\usepackage{amsthm} 
\usepackage{amsmath} 
\usepackage{tikz-cd}
\usepackage{float}
\usepackage[]{hyperref}
\usepackage{minted}
\hypersetup{
  colorlinks,
  linkcolor=blue,
  citecolor=blue,
  urlcolor=blue}
\newcommand{\hask}[1]{\mintinline{Haskell}{#1}}
\newenvironment{haskell}
  {\VerbatimEnvironment
  	\begin{minted}[escapeinside=??, mathescape=true, tabsize=1, bgcolor=black!3, xleftmargin=3pt ]{Haskell}}
  {\end{minted}}
\makeatletter
% replace \medskip before and after the box with nothing, i.e., remove it
\patchcmd{\minted@colorbg}{\medskip}{}{}{}
\patchcmd{\endminted@colorbg}{\medskip}{}{}{}
\makeatother

\author{Bartosz Milewski}
\title{Actegories and Lenses}
\begin{document}
\maketitle{}

\section{Motivation}
A lens is a reification of the concept of object composition. In a nutshell, it describes the process of decomposing the source object $s$ into a focus $a$ and a residue $c$ and recomposing a new object $t$ from a new focus $b$ and the same residue $c$. 

The key observation is that we don't care what the residue is as long as it exists. This is why a lens can be implemented, in Haskell, as an existential type:
\begin{haskell}
data Lens s t a b where
 Lens :: (s -> (c, a)) -> ((c, b) -> t) -> Lens s t a b
\end{haskell}
In category theory, the existential type is represented as a coend---essentially a gigantic sum over all objects $c$ in the category $\mathcal{C}$:
\[ \mathcal{L}\langle s, t\rangle \langle a, b \rangle = \int^{c \colon \mathcal{C}} \mathcal{C}(s, c \times a) \times  \mathcal{C}(c \times b, t) \]

There is a simple recipe to turn this representation into the more familiar one. The first step is to use the currying adjunction on the second hom-functor:
\[  \mathcal{C}(c \times b, t) \cong  \mathcal{C}(c, [b, t]) \]
Here, $[b, t]$ is the internal hom, or the function object \hask{b->t}.
Once the object $c$ appears as the source in the hom-set, we can use the co-Yoneda lemma to eliminate the coend. This is the formula we use:
\[ \int^c F c \times  \mathcal{C}(c, x) \cong F x\]
It works for any functor $F$ from the category $\mathcal{C}$ to $\mathbf{Set}$ so, in particular we have:
\[ \mathcal{L}\langle s, t\rangle \langle a, b \rangle \cong \int^{c} \mathcal{C}(s, c \times a) \times  \mathcal{C}(c, [b, t]) \cong \mathcal{C}(s, [b, t] \times a) \]
The result is a pair of arrows:
\[ \mathcal{C}(s, [b, t]) \times \mathcal{C}(s, a) \]
the first corrsponding to:
\begin{haskell}
set :: s -> b -> t
\end{haskell}
and the second to:
\begin{haskell}
get :: s -> a
\end{haskell}

It turns out that this trick works for more general optics. It all depends on being able to isolate the object $c$ as the source of the second hom-set. 

We've been able to do it case-by-case for lenses, prisms, traversals, and the whole zoo of optics. It turns out that the same problem was studied in all its generality by Australian category theorists \href{http://www.tac.mta.ca/tac/volumes/9/n4/n4.pdf}{Janelidze and Kelly} in a context that had nothing to do with optics.

\section{Monoidal Actions}

Here's the most general definition of an optic:
\[ \mathcal{O}\langle s, t\rangle \langle a, b \rangle = \int^{c \colon \mathcal{M}} \mathcal{D}(s, K_c a) \times  \mathcal{C}(L_c b, t) \]
This definition involves three separate categories. The category $\mathcal{M}$ is monoidal, and it has an action defiend on both $\mathcal{C}$ and $\mathcal{D}$. A category with a monoidal action is called an \emph{actegory}.

We get the definition of a lens by having all three categories be the same---a cartesian closed category $\mathcal{C}$. We define the action of this category on itself:
\[L_c a = c \times a \]
(and the same for $K_c$). 

There are two equivalent ways of defining the action of a monoidal category. One is as the mapping 
\[ \bullet \colon \mathcal{M} \times \mathcal{C} \to \mathcal{C} \]
written in infix notation as $c \bullet a$. It has to be equipped with two additional structure maps---isomorphisms that relate the tensor product $\otimes$ in $\mathcal{M}$ and its unit $I$ to the action in $\mathcal{C}$:

\[ \alpha_{d c a} \colon (d \otimes c) \bullet a  \to d \bullet (c \bullet a) \]
\[ \lambda_a \colon I \bullet a \to a \]
plus some coherence conditions corresponding to associativity and unit laws. 

Using this notation, we can rewrite the definition of an optic as:
\[ \mathcal{O}\langle s, t\rangle \langle a, b \rangle = \int^{c \colon \mathcal{M}} \mathcal{D}(s, c \bullet a) \times  \mathcal{C}(c \bullet b, t) \]
with the understanding that we use the same symbol for two different actions in $\mathcal{C}$ and $\mathcal{D}$.

Alternatively, we can curry the action $\bullet$, and use the mapping:
\[ L \colon \mathcal{M} \to [\mathcal{C}, \mathcal{C}] \]
The target category here is the category of endofunctors $[\mathcal{C}, \mathcal{C}]$, which is naturally equipped with a monoidal structure given by functor composition (and, as we well know, a monad is just a monoid in that category). 

The two structure maps from the definition of $\bullet$ translate to the requirement that $L$ be a strict monoidal functor, mapping tensor product to functor composition and unit object to identity functor.

\section{The Adjunction}

When we were eliminating the coend in the definition of a lens, we used the currying adjunction. This particular adjunction works inside a single category but, in general, an adjunction relates two functors between a pair of categories. Therefore, to eliminate the end from the optic, we need an adjunction that looks like this:
\[ \mathcal{C}(c \bullet a, t) \cong \mathcal{M}(c, R_a t) \]
The category on the right is the monoidal category $\mathcal{M}$, because $c$ is the object from this category. 

Using the adjunction and the co-Yoneda lemma we get:
\[ \mathcal{O}\langle s, t\rangle \langle a, b \rangle = \int^{c \colon \mathcal{M}} \mathcal{D}(s, c \bullet a) \times  \mathcal{M}(c, R_b t) \cong  \mathcal{D}(s, R_b t \bullet a) \]


There is a whole slew of monoidal actions that have the right adjoint of this type. Let's look at an example.

The category of sets is a monoidal category, and we can define its action on another category using the formula:
\[ \mathcal{C}(c \cdot a, t) \cong \mathbf{Set}(c, \mathcal{C}(a, t)) \]
This is the definition of a \emph{copower}. A category in which this adjunction holds for all objects is called \emph{copowered} or \emph{tensored} over $\mathbf{Set}$. 

The intuition is that a copower is like an iterated sum (hence the multiplication sign). Indeed, a mapping out of a coproduct of $c$ copies of $a$, where $c$ is a set, is equivalent to $c$ mappings out of $a$. 

This formula generalizes to the case in which $\mathcal{C}$ is a category enriched over a monoidal category $\mathcal{V}$. In that case we have:
\[ \mathcal{C}(c \cdot a, t) \cong \mathcal{V}(c, \mathcal{C}(a, t)) \]
Here, the hom $\mathcal{C}(a, t)$ is an object in $\mathcal{V}$.

\section{Relation to Enrichment}

We are interested in the adjunction:
\[ \mathcal{C}( c \bullet a, t) \cong \mathcal{M}(c, R_a t) \]
The functor $R$ is covariant in $t$ and contravariant in $a$:
\[ R \colon \mathcal{C}^{op} \times \mathcal{C} \to \mathcal{M} \]
In other words, it's a profunctor. In fact, it has the right signature to be a hom-functor. And this is what Janelidze and Kelly show: the functor $R$ can serve as the hom-functor that generates the enrichment for $\mathcal{C}$. If we call the enriched category $\mathbf{C}$, we can define the hom-object as:
\[\mathbf{C}(a, t) = R_a t \]
Our adjunction can be rewritten as:
\[ \mathcal{C}( c \bullet a, t) \cong \mathcal{M}(c, \mathbf{C}(a, t)) \]
The counit of this adjunction is a mapping:
\[ \epsilon_{a t} \colon \mathbf{C}(a, t)  \bullet a \to t \]
which is analogous to function application.

The hom-object $\mathbf{C}(a, t)$ in an enriched category must satisfy the composition and identity laws. In an enriched category, composition is a mapping:
\[ \circ \colon \mathbf{C}(b, t) \otimes \mathbf{C}(a, b) \to \mathbf{C}(a, t) \]
Let's see if we can get this mapping from our adjunction by replacing $c$ with $\mathbf{C}(b, t) \otimes \mathbf{C}(a, b)$. We get:
\[ \mathcal{C}( (\mathbf{C}(b, t) \otimes \mathbf{C}(a, b)) \bullet a, t) \cong \mathcal{M}(\mathbf{C}(b, t) \otimes \mathbf{C}(a, b), \mathbf{C}(a, t)) \]
The right-hand side should contain the mapping we're looking for. All we need is to point at a morphism on the left. Indeed, we can define it as the following composite:
\begin{align*}
\big( & \mathbf{C}(b, t) \otimes \mathbf{C}(a, b)\big) \bullet a  \xrightarrow{\alpha}
\\ &\mathbf{C}(b, c) \bullet \big(\mathbf{C}(a, b)) \bullet a\big)   \xrightarrow{id \; \bullet \; \epsilon}
\\ &\mathbf{C}(b, t) \bullet b  \xrightarrow{\epsilon} t
\end{align*}
We used the structure map $\alpha$ and (twice) the counit of the adjunction $\epsilon$.

Similarly, the identity of composition, which is the mapping:
\[ id_a \colon I \to \mathbf{C}(a, a) \]
is adjoint to the other structure map $\lambda_a$.

Janelidze and Kelly go on to prove that the action of a monoidal right-closed category having the right adjoint is equivalent to the existence of the tensored enrichment of the category on which this action is defined.

The two examples of monoidal actions we've seen so far are indeed equivalent to enrichments. A cartesian closed category in which we defined the action $L_c a = c \times a$ is automatically self-enriched. The copower action $L_c a = c \cdot a$ is equivalent to enrichment over $\mathbf{Set}$ (which doesn't mean much, since regular categories are naturally $\mathbf{Set}$-enriched; but not all of them are tensored).

\section{Acknowledgments}
I'm grateful to Jules Hedges and his collaborators for sharing their insights and unpublished notes with me.

\end{document}